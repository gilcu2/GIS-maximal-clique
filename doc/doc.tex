% Created 2013-11-12 Tue 20:00
\documentclass[12pt, a4paper]{article}
\usepackage{polski}
\usepackage[utf8]{inputenc}
\usepackage[polish]{babel} 
\usepackage{geometry}
\usepackage{hyperref}
\usepackage{amsmath}
\usepackage[numbers]{natbib}
\author{Marek Lewandowski, Juliusz Gonera}
\date{}
\title{Porównanie algorytmów znajdowania maksymalnej kliki w grafie}
\begin{document}

\maketitle

\section{Problem maksymalnej kliki}
\label{sec-1}
Kliką nazywamy spójny podgraf, taki że nie jest on zawarty w żadnym innym spójnym podgrafie. Maksymalna klika to klika składająca się z największej liczby wierzchołków. Problem znajdowania maksymalnej kliki w grafie jest problemem NP-zupełnym.

\section{Algorytm}
\label{sec-2}
Do zaimplementowania został wybrany algorytm przechodzący graf w głąb i używający techniki branch-and-bound w celu znalezienia maksymalnej kliki. Algorytm został opisany w \citep{bioinf} (Fig. 2). Zostanie on porównany z dostępną implementacją algorytmu Brona-Kerboscha w bibliotece JGraphT\citep{jgrapht}.

\section{Technologia}
\label{sec-3}
Program zostanie zaimplementowany w języku Scala. Jest to statycznie typowany język działający na JVM, w pełni kompatybilny z językiem Java.

\section{Testy}
\label{sec-4}
Do testowania poprawności i empirycznego porównania wydajności algorytmów zostanie użyty DIMACS Benchmark Set\citep{dimacs}. Jest to zbiór nietrywialnych grafów dla których znana jest liczba wierzchołków tworzących maksymalną klikę w grafie.

\nocite{*}
\bibliographystyle{plainnat}
\bibliography{bibliography}
\end{document}
