% Created 2013-11-12 Tue 19:43
\documentclass[11pt]{article}
\usepackage[utf8]{inputenc}
\usepackage[T1]{fontenc}
\usepackage{fixltx2e}
\usepackage{graphicx}
\usepackage{longtable}
\usepackage{float}
\usepackage{wrapfig}
\usepackage[normalem]{ulem}
\usepackage{textcomp}
\usepackage{marvosym}
\usepackage{wasysym}
\usepackage{latexsym}
\usepackage{amssymb}
\usepackage{amstext}
\usepackage{hyperref}
\tolerance=1000
\usepackage{biblatex}
\bibliography{bibliography}
\author{Marek Lewandowski, Juliusz Gonera}
\date{}
\title{Porównanie algorytmów znajdowania maksymalnej kliki w grafie}
\hypersetup{
  pdfkeywords={},
  pdfsubject={},
  pdfcreator={Emacs 24.3.1 (Org mode 8.0.5)}}
\begin{document}

\maketitle

\section{Problem maksymalnej kliki}
\label{sec-1}
Kliką nazywamy spójny podgraf, taki że nie jest on zawarty w żadnym innym spójnym podgrafie. 
Maksymalna klika to klika składająca się z największej liczby wierzchołków. 
Problem znajdowania maksymalnej kliki w grafie jest problemem NP-zupełnym.

\section{Algorytm}
\label{sec-2}
Do zaimplementowania został wybrany algorytm przechodzący graf w głąb i używający techniki branch-and-bound w celu znalezienia maksymalnej kliki. Algorytm został opisany w \cite{bioinf} (Fig. 2). Zostanie on porównany z dostępną implementacją algorytmu Brona-Kerboscha w bibliotece JGraphT\cite{jgrapht}.

\section{Technologia}
\label{sec-3}
Program zostanie zaimplementowany w języku Scala. Jest to statycznie typowany język działający na JVM, w pełni kompatybilny z językiem Java.

\section{Testy}
\label{sec-4}
Do testowania poprawności i empirycznego porównania wydajności algorytmów zostanie użyty DIMACS Benchmark Set\cite{dimacs}. Jest to zbiór nietrywialnych grafów dla których znana jest liczba wierzchołków tworzących maksymalną klikę w grafie.

\printbibliography
% Emacs 24.3.1 (Org mode 8.0.5)
\end{document}
