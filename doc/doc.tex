% Created 2013-11-12 Tue 20:00
\documentclass[12pt, a4paper]{article}
\usepackage{polski}
\usepackage[utf8x]{inputenc}
\usepackage[polish]{babel} 
\usepackage{geometry}
\usepackage{hyperref}
\usepackage{amsmath}
\usepackage[numbers]{natbib}
\usepackage{algorithm}
\usepackage{algpseudocode}
\usepackage{float}
\usepackage{graphicx}
\usepackage{listings}

\lstdefinelanguage{scala}{
  morekeywords={abstract,case,catch,class,def,%
    do,else,extends,false,final,finally,%
    for,if,implicit,import,match,mixin,%
    new,null,object,override,package,%
    private,protected,requires,return,sealed,%
    super,this,throw,trait,true,try,%
    type,val,var,while,with,yield},
  otherkeywords={=>,<-,<\%,<:,>:,\#,@},
  sensitive=true,
  morecomment=[l]{//},
  morecomment=[n]{/*}{*/},
  morestring=[b]",
  morestring=[b]',
  morestring=[b]"""
}

\author{Marek Lewandowski, Juliusz Gonera}
\date{}
\title{Porównanie algorytmów znajdowania maksymalnej kliki w grafie}
\begin{document}

\maketitle

\section{Problem maksymalnej kliki}
\label{sec-1}
Definicje klik:
\begin{description}
\item[Klika] - kliką w grafie $G$ jest kompletny podgraf grafu $G$.
\item[Klika maksymalna] - klika, która nie może być rozszerzona poprzez dołączenie kolejnego sąsiadującego wierzchołka - oznacza to, że klika maksymalna nie jest podzbiorem innej większej kliki. 
\item[Największa klika] - klika o największej ilości wierzchołków. Największa klika jest kliką maksymalną, ale odwrotne stwierdzenie nie jest prawdziwe.
\end{description}

Definicje klik zostały zaczerpnięte z \cite{weissteinClique}. Problem znajdowania maksymalnej kliki w grafie jest problemem NP-zupełnym i jest jednym z 21 NP zupełnych problemów Karpa \cite{Kar72}.

Powyższa terminologia w praktyce jest stosowana trochę skrótowo. Mianowicie szukanie maksymalnej kliki odnosi się do poszukiwań kliki największej, która z definicji jest także maksymalna. Natomiast szukanie wszystkich maksymalnych klik odnosi się do problemu, w których szukanych jest wiele klik o różnych rozmiarach, z których wszystkie są maksymalne, ale nie wszystkie są największe. Przykładowo w grafie o trzech składowych K3, K3 i K4 istnieją 3 kliki maksymalne, ale tylko jedna z nich jest największa - ta o rozmiarze 4. 

W poniższej pracy mówiąc o maksymalnej klice odnosimy się do największej maksymalnej kliki. 

\section{Algorytm BasicMC}
\label{sec-2}
Większość informacji zawartych w niniejszym rozdziale jest taka sama w stosunku do wstępnego sprawozdania. Zaznajomiony czytelnik może śmiało przejść do opisu algorytmu Brona-Kerboscha w rozdziale \ref{bron-kerbosch-alg}.

Do zaimplementowania został wybrany algorytm \ref{basicmc} przechodzący graf w głąb i używający techniki branch-and-bound w celu znalezienia maksymalnej kliki. Algorytm ten został wybrany z powodu jego elegancji i prostoty. Algorytm został opisany w \citep{bioinf} (Fig. 2). Zostanie on porównany z dostępną implementacją algorytmu Brona-Kerboscha w bibliotece JGraphT\citep{jgrapht}.

\subsection{Struktury danych}

W algorytmie \ref{basicmc} wykorzystywane są zbiory $Q$ i $Q_{max}$. $Q$ przechowuje wierzchołki aktualnie rozpatrywanej kliki. $Q_{max}$ zawiera wierzchołki największej kliki jaką dotąd udało się znaleźć. $R \subseteq V $ zawiera listę wierzchołków które mogą zostać dodane do $Q$.

\subsection{Opis Algorytmu}

Początkowo $Q := \emptyset, Q_{max} := \emptyset, R := V$. Wybieramy wierzchołek $p$ z dostępnych wierzchołków w $R$ i dodajemy go do $Q$ \algref{basicmc}{addPToQ}. Następnie obliczamy $R_{p} := R \cap \text{adj(}p\text{)}$ który staje się nowym zbiorem wierzchołków do przejrzenia. Procedura EXPAND() jest wywoływana rekursywnie do wyczerpania zbioru wierzchołków $R_{p}$. Jeśli $|Q|+|R| \leq |Q_{max}|$ to $Q \cup R$ może zawierać jedynie klikę niewiększą od $|Q_{max}|$, w tym przypadku algorytm pomija zbędne obliczenia, ponieważ wiadomo, że nie znajdzie tam większej kliki.\algref{basicmc}{skip}

\subsection{Warunek Stopu}

W momencie osiągnięcia $R_{p} = \emptyset$, Zbiór $Q_{max}$ jest maksymalną kliką. Jeśli $|Q| > |Q_{max}|$ zbiór $Q$ zastępuje zbiór $Q_{max}$. Po usunięciu wierzchołka początkowego $p$ z $Q$ i $R$ wybieramy go z wierzchołków pozostałych w $R$ i powtarzamy do przejrzenia wszystkich wierzchołków ($R = \emptyset$)

\subsection{Złożoność Obliczeniowa}
W grafie o $n$ wierzchołkach może być nie więcej niż $3^{\frac{n}{3}}$ maksymalnych klik. Złożoność obliczeniowa algorytmu Brona-Kerboscha wynosi $O(3^{\frac{n}{3}})$.
Jest to także dolne ograniczenie dla złożoności obliczeniowej implementowanego algorytmu. Faktyczne zachowanie algorytmu w zależności od rozmiaru grafu wejściowego zostało wyznaczone empirycznie.

\subsection{Złożoność pamięciowa}
\label{memory_complexity}

Złożoność pamięciowa implementowanego algorytmu to $O(|V|)$. Ilość pamięci potrzebnej jest wprost proporcjonalna do liczby wierzchołków grafu. Wynika to bezpośrednio z definicji i sposobu użycia zbiorów $Q$, $Q_{max}$ i $R$.

\section{Algorytm Brona-Kerboscha}
\label{bron-kerbosch-alg}
Zaimplementowany algorytm BasicMC opisany powyżej został porównany z klasycznym algorytmem Brona-Kerboscha, a konkretnie z jego implementacją udostępnianą przez bibliotekę JGraphT.

<<<<<<< HEAD
\subsection{Opis Algorytmu}

Algorytm Brona-Kerboscha wybiera wierzchołek $v$ z $P$. Dodaje $v$ do $R$ i usuwa wierzchołki nie incydentne z v ze zbiorów $P$ i $X$. Dalej wybieramy kolejny wierzchołek z nowego zbioru $P$ i powtarzamy proces. Kontynuujemy dopóki $P$ nie jest zbiorem pustym. Gdy $P$ jest pusty i $X$ jest pusty zgłaszamy znalezienie nowej największej kliki (jeśli nie jest największa to $R$ zawiera podzbiór już znalezionej kliki), zawartej w zbiorze $R$. Wtedy cofamy się do ostatniego wyberanego wierzchołka i przywracamy $P$, $R$, $X$ do stanu przed wyborem kliki, usuwamy wierzchołek z $P$, dodajemy go do $X$ i rozwijamy kolejny wierzchołek.\citep{bron-kerbosch}
=======
Algorytm Brona-Kerboscha wybiera wierzchołek $v$ z $P$. Dodaje $v$ do $R$ i usuwa wierzchołki nieincydentne z v ze zbiorów $P$ i $X$. Dalej wybieramy kolejny wierzchołek z nowego zbioru $P$ i powtarzamy proces. Kontynuujemy dopóki $P$ nie jest zbiorem pustym. Gdy $P$ jest pusty i $X$ jest pusty zgłaszamy znalezienie nowej największej kliki (jeśli nie jest największa to $R$ zawiera podzbiór już znalezionej kliki), zawartej w zbiorze $R$. Wtedy cofamy się do ostatniego wybranego wierzchołka i przywracamy $P$, $R$, $X$ do stanu przed wyborem kliki, usuwamy wierzchołek z $P$, dodajemy go do $X$ i rozwijamy kolejny wierzchołek.\citep{bron-kerbosch}
>>>>>>> Be explicit about repeated paragraphs from previous docs. Improvements here and there.

\subsection{Złożoność}

Jak dotąd nie zostały opublikowane teoretyczne ograniczenia dla złożoności klasycznego algorytmu Brona-Kerboscha\citep{tomita}. Empiryczne rezultaty z naszy testów zostały zamieszczone w sprawozdaniu\ref{fig:time-complexity}. Istnieją odmiany tego algorytmu działające w czasie $O(3^{n/3})$\citep{tomita}.

W ogólności problem znajdowania kliki jest NP trudny\citep{karp}. Algorytm o najniższym ograniczeniu teoretycznym cechuje się złożonością czasową należącą do klasy $O(1.8888^{n})$\citep{robson}

\section{Różnice między algorytmami}
Zasadniczą różnicą między algorytmem BasicMC i algorytmem Brona-Kerboscha jest to, że BasicMC szuka jedynie największej kliki. Algorytm Brona-Kerboscha znajduje wszystkie maksymalne kliki, z których później wybierana jest największa.

TODO czy coś więcej można tutaj napisać?

\begin{algorithm}[H]
\caption{BasicMC}\label{basicmc}
\begin{algorithmic}[1]
  
\Procedure{BasicMC}{$V,E$}
\State $Q\gets \emptyset$;
\State $Q_{max}\gets \emptyset$;
\State \Call{EXPAND}{V};
\EndProcedure
\Statex

\Procedure{expand}{$R$}
\While{$R \not= \emptyset$}
  \State $p\gets v\in R$
  \If{$|Q|+|R| > |Q_{max}|$}
    \State $Q \gets Q \cup {p}$\label{addPToQ}
    \State $R_p \gets R \cap adj(p)$
    \If{$R_p \not= \emptyset$}
      \State \Call{EXPAND}{$R_{P}$}
    \ElsIf{$|Q| > |Q_{max}|$}
      \State $Q_{max} \gets Q$
    \EndIf
    \State $Q \gets Q - {p}$
  \Else
    \textbf{ return}\label{skip}
  \EndIf
  \State $R \gets R - p$
\EndWhile
\EndProcedure

\end{algorithmic}
\end{algorithm}

\begin{algorithm}[H]
\caption{Bron-Kerbosch}\label{bron-kerbosch}
\begin{algorithmic}[1]
  
\Procedure{BronKerbosch}{$R,P,X$}
  \If{$P$ and $Q$ are both empty}
    \State report $R$ as maximal clique
  \EndIf
  \ForAll{$v \in P$}
    \State BronKerbosch($R \cup v, P \cap adj(v), X \cap adj(v)$)
    \State $P \gets P - v$
    \State $X \gets X \cup v$
  \EndFor
\EndProcedure

\end{algorithmic}
\end{algorithm}


\section{Projekt testów}
\label{sec-4}
Projekt testów został opisany w dokumentacji wstępnej i nie uległ gruntownym zmianom. Jedynymi zmianami było wykonanie większej ilości testów funkcjonalnych oraz odrzucenie rozgrzewania JVM ze względu na czynniki opisane w rozdziale porównującym algorytmy \ref{algorithm-comparison}. Poniżej znajdują się te same informacje, które zaznajomiony czytelnik może pominąć przechodząc do rozdziału \ref{artifacts}.

\subsection{Podział testów}
Projekt testów został podzielony na 3 części, każda odpowiedzialna za inny aspekt programu.

\subsection{Testy funkcjonane}
Mają na celu pokazanie, że zaimplementowany algorytm jak i reszta aplikacji pozbawiona jest błędów. W ramach tych testów przetestowane zostaną przypadki:

\begin{itemize}
\item grafu pustego z różną liczbą wierzchołków - maksymalna klika powinna wynosić $1$
\item grafu pełnego $K_{n}$ - maksymalna klika wynosi $n$
\item grafu niespójnego zawierającego podgrafy $K_{n_{1}}$ $K_{n_{2}}$ - maksymalna klika wynosi $max(n_{1}, n_{2})$
\item grafu, które jest drzewem - maksymalna klika powinna wynosić $2$
\item kół $W_{n}$ - maksymalna klika powinna wynosić $3$
\end{itemize}

Dla pozostałej części programu - obsługi wejścia, wyjścia, opcji i innych - zostaną wykonane odpowiednie testy.

\subsection{Testy złożoności}
\subsubsection{Złożoność pamięciowa}
Złożoność pamięciowa zostanie ustalona empirycznie na podstawie pomiarów pamięci zużytej przez program dla grafów o różnej wielkości. Oczekiwana złożoność pamięciowa programu to $O(|V|)$ (\ref{memory_complexity}). 

\subsubsection{Złożoność obliczeniowa}
Testy mają na celu empiryczne zmierzenie złożoności obliczeniowej. W tym celu zostaną wygenerowane grafy losowe z różnym prawdopodobieństwem krawędzi i różną liczbą wierzchołków. Dla każdego z wygenerowanych grafów uruchomiony zostanie algorytm. Zebrane wyniki poddane zostaną analizie statystycznej, która pozwoli określić złożoność obliczeniową.

\subsection{Testy porównawcze}
W celu empirycznego porównania wydajności algorytmów zostanie użyty DIMACS Benchmark Set\citep{dimacs}. Jest to zbiór nietrywialnych grafów 
dla których znana jest liczba wierzchołków tworzących maksymalną klikę w grafie.

\section{Artefakty projetkowe}
\label{artifacts}
Razem ze sprawozdaniem dostarczone zostały następujące aretefakty projektowe:
\begin{itemize}
\item Gotowy wykonywalny program gis-maximal-clique.jar
\item Kod źródłowy programu
\item Plik $\text{dimacs-best-90sec}$ z wynikami z uruchomienia programu na grafach ze zbioru DIMACS 
\item Plik $\text{benchmark-50-0.7.txt}$ z wynikami z testowania złożoności pamięciowej i czasowej
\item Automatycznie wygenerowana dokumentacja dla programisty
\end{itemize}

\section{Instrukcja obsługi programu}
Poniższa instrukcja powinna w kompletny sposób zapoznać użytkownika z tym jak program uruchamiać i jakie ma możliwości. Dostarczony program to typowa aplikacja CLI. Program wykonuje się na wirtualnej maszynie Javy. Program uruchamiany był na JVM w wersji 1.7. Program należy uruchamiać z linii poleceń w standardowy sposób dla programów Javowych, mianowicie:

\begin{verbatim}
java -jar gis-maximal-clique.jar
\end{verbatim}

Jako, że niektóre grafy są dość duże warto zwiększyć maksymalną ilość pamięci dostępną dla programu. 2G pamięci były wystarczające dla większości grafów ze zbioru DIMACS. Należy wtedy uruchomić program w następujący sposób:

\begin{verbatim}
java -Xmx2G -jar gis-maximal-clique.jar
\end{verbatim}

\subsection{Dane wejściowe}
Program oczekuje danych na standardowym wejściu. Danymi jest graf w formacie DIMACS. W środowisku linux korzystamy z operatora przekierowania w następujący sposób:

\begin{verbatim}
java -Xmx2G -jar gis-maximal-clique.jar < graf
\end{verbatim}

\subsubsection{Format DIMACS}

Wejściem programu są pliki tekstowe ASCII. Wejście zawiera $|E|+1$ linii nie licząc linii zawierających komentarzy. Pierwsza linijka nie będąca komentarzem jest postaci 
\begin{verbatim}
p col |V| |E|
\end{verbatim}
Gdzie $\text{p col}$ to po prostu przyjęte znaki tekstowe. $|V|$ i $|E|$ to odpowiednio liczba węzłów i krawędzi grafu. Następne $|E|$ linijek odpowiada $|E|$ krawędziom grafu. Krawędź $(v, w)$ zapisywana jest jako 
\begin{verbatim}
e W V
\end{verbatim}
i występuje tylko raz. Krawędź $(w, v)$ stanowi części reprezentacji tekstowej grafu. Podobnie jak wcześniej $e$ to zwykły znak. Opisany format jest podzbiorem formatu DIMACS opisanego w \cite{dimacs_format}

\subsection{Opcje programu}
Dostępne opcje można zobaczyć poprzez uruchomienie programu bez żadnych dodatkowych argumentów i bez podawania danych wejściowych. Opcje opisane zostały także poniżej.

\subsection{Wybór algorytmu}
Domyślnie program korzysta z algorytmu BasicMC. Aby skorzystać z algorytmu Brona-Kerboscha należy dodać argument $-j$

\subsection{Ograniczenia czasowe}
Domyślnie program działa bez ograniczenia czasowego. Mając na uwadze złożoność problemu, program może wykonywać się bardzo długo szczególnie dla dużych grafów. Można ograniczyć działanie algorytmu poprzez podanie argumentu \emph{\text{-max sekundy}} podając odpowiednią liczbę sekund.

Ograniczenie to dotyczy czasu działania \emph{algorytmu}, a nie \emph{programu}. Dla dużych grafów i krótkich czasów budowanie grafu może zająć dużo czasu, znacznie więcej niż podane ograniczenie. Ograniczenie dotyczy tylko działania algorytmu, a zatem program zakończy się po czasie nie większym niż potrzebnym na załadowanie i zbudowanie grafu dodając do tego ograniczenie w sekundach.

\subsection{Wyniki cząstkowe}
Domyślnie program wyświetla tylko najlepsze znalezione rozwiązanie. Korzystając z opcji $-progress$ program będzie wyświetlał na bieżąco kolejne najlepsze rozwiązania.

\subsection{Tylko rozmiar kliki}
Domyślnie program wyświetla etykiety wierzchołków tworzących klikę. Korzystając z opcji $-sizeOnly$ program będzie wyświetlał tylko rozmiar kliki bez wypisywania wierzchołków. Więcej informacji przy opisie formatowania.

\subsection{Formatowanie}
Domyślnie program wyświetla wyniki w poniższym formacie:

\begin{verbatim}
w(g) TIME
V1 V2 V3 ...
\end{verbatim}
Gdzie $\omega(g)$ to rozmiar największej znalezionej kliki, $\text{TIME}$ - czas wykonania algorytmu w milisekundach, $\text{V1 V2 V3}$ to etykiety wierzchołków, które tworzą klikę maksymalną. Przykładowy wynik to:

\begin{verbatim}
24 613
101 120 52 110 125 20 93 29 121 85 105 17 118 12 81 39 98 66 3 67 11 23 119 47
\end{verbatim}

Korzystając z opcji $-sizeOnly$ format wyjściowy nie będzie zawierał drugiej linijki. 

Korzystając z opcji $-csv$ program będzie wyświetlał wynik w formacie CSV z większa ilością informacji. Kolejne kolumny to:
\begin{verbatim}
Nazwa grafu, Użyty algorytm, w(g), TIME, MEMORY, CLIQUE
\end{verbatim}

gdzie $\text{MEMORY}$ to zużyta pamięć w kB, a $\text{CLIQUE}$ to numery wierzchołków podobnie jak w prostej wersji formatu. Korzystając z opcji $-sizeOnly$ wyjście nie będzie zawierało kolumny $\text{CLIQUE}$.

\subsection{Tryb benchmark}
Program można włączyć w trybie benchmark. Będzie wtedy testował dany algorytm na losowo wygenerowanym spójnym grafie nieskierowanym z zadanymi parameterami. Aby włączyć program w tym trybie należy podać argument z dwiema wartościami \emph{\text{-benchmark N p}} gdzie $N$ to maksymalna liczba wierzchołków w losowo wygenerowanym grafie, a $p$ to prawdopodobieństwo wystąpienia krawędzi. $N$ powinno być większe od 10.

Opcja benchmark łączy się tylko z wyborem algorytmu. Pozostałe argumenty są pomijane. Format wyjściowy to CSV.

\subsubsection{Działanie trybu benchmark}
W tym trybie wybrany algorytm wykonywany jest na coraz większych grafach losowych. Zaczynając od 10 wierzchołków, aż do podanego w opcji $N$ budowane są losowe grafy z prawdopodobieństwem krawędzi $p$. Na grafie mierzony jest czas wykonania oraz zużyta pamięć dla obydwu algorytmów. Dla każdego pośredniego rozmiaru algorytm wykonywany jest kilkukrotnie. Wyniki są agregowane i uśredniane. Następnie zaczyna się testowanie na większym losowym grafie, aż do osiągnięcia podanego $N$.

\subsection{Przykłady}

\begin{verbatim}
java -Xmx2G -jar gis-maximal-clique.jar -max 90 -progress -csv < data/graph
\end{verbatim}
Uruchamia program z ograniczeniem czasowym 90 sekund korzystając z algorytmu BasicMC wyświetlając wyniki pośrednie w formacie CSV dla grafu w formacie dimacs dostępnego pod ścieżką data/graph.

\begin{verbatim}
java -Xmx2G -jar gis-maximal-clique.jar -max 30 -j < data/graph 
\end{verbatim}
Uruchamia program z ograniczeniem czasowym 30 sekund korzystając z algorytmu Brona-Kerboscha dla grafu w formacie dimacs dostępnego pod ścieżką data/graph

\begin{verbatim}
java -Xmx2G -jar gis-maximal-clique.jar -benchmark 60 0.9
\end{verbatim}
Uruchamia program w trybie benchmark na losowych grafach zaczynając od liczby wierzchołków 10 aż do 60 z prawdopodobieństwem wystąpienia krawędzi 0.9 korzystając z algorytmu BasicMC.

\begin{verbatim}
java -Xmx2G -jar gis-maximal-clique.jar  -benchmark 50 0.7 -j
\end{verbatim}
Uruchamia program w trybie benchmark na losowych grafach zaczynając od liczby wierzchołków 10 aż do 50 z prawdopodobieństwem wystąpienia krawędzi 0.7 korzystając z algorytmu Brona-Kerboscha.

\section{Wprowadzenie do kodu dla programisty}
Program został napisany w języku Scala. Do sprawdozdania dołączona została automatycznie wygenerowania dokumentacja ScalaDoc, która jest bardzo podobna do JavaDocu. Poniżej zebrane zostały najważniejsze informacje o programie służące do wprowadzenia programisty w temat programu.

Program został podzielony na dwa moduły. Moduł IO i moduł grafów. 

\subsection{Moduł IO}
Moduł IO odpowiada za realizację funkcjonalności typowej dla aplikacji CLI, czyli interpretację argumentów, odczytanie wejścia, delegacji odpowiednich akcji do logiki grafów w zależności od podanych argumentów i wypisanie wyników na wyjście. Moduł IO został zdefiniowany w pakiecie \emph{app}.

\subsection{Moduł grafów}
Moduł grafów zawiera definicje generycznego funkcyjnego\footnote{a zatem niemutowalnego} grafu, implementacje niezorientowanego grafu, krawędzi, wierzchołków, algorytmów i funkcji pomocniczych. Najważniejszą funkcją jest funkcja \text{{graphs.Graph.findBiggestClique}}, która dokonuje obliczeń na grafie w wątkach działających w tle i w sposób asynchroniczny zwraca kolejne najlepsze wyniki. Użyta została tutaj abstrakcja znana z funkcyjnego reaktywnego programowania \emph{Observable}. Ze szczegółami tej abstrakcji czytelnik może zapoznać się u źródła \cite{rx}

\subsubsection{Modyfikacje biblioteki JGraphT}
W celu realizacji funkcjonalności wyników cząstkowych kod biblioteki musiał zostać zmodyfikowany. Konkretna klasa \emph{BronKerboschCliqueFinder} nie była otwarta na modyfikacje, dlatego też została przekopiowana do źródeł i zmodyfikowana. Dodana została także odpowiednia klasa w Scali opakowująca rozszerzoną klasę Javową z biblioteki tak, aby API było spójne i naturalne dla reszty programu.
Modyfikacje biblioteki znajdują się w pakiecie \emph{extensions}.

\subsection{Modyfikacje kodu}
\begin{itemize}
\item Nowe opcje, formaty wyjściowe i tym podobne powinny być dodawane do klasy \emph{app.App}.
\item Nowe algorytmy powinny być dodawane do klasy \emph{graphs.Graph}
\item Rozszerzenia do danych przechowywanych w wierzchołkach i krawędziach należy wprowadzać w odpowiednich klasach z pakietu \emph{graphs}
\end{itemize}

\subsection{API}
Szczegóły dotyczące API zawarte są w dokumentacji ScalaDoc dołączonej do sprawozdania.

\newpage
\subsection{Implementacja algorytmu BasicMC}
Dzięki bogatej składni języka Scala implementacja wygląda bardzo podobnie do pseudokodu algorytmu.

\begin{lstlisting}[language=scala]  % Start your code-block

def maximalClique(g: UndirectedGraph): Set[Node] = {
    type V = Node
    var Q = Set[V]()
    var Qmax = Set[V]()

    def expand(r: Set[V]): Set[V] = {
      var R = r
      while (R.nonEmpty) {
        val p = R.head
        if (Q.size + R.size > Qmax.size) {
          Q = Q + p
          val Rp = R intersect g.adj(p)
          if (Rp.nonEmpty) expand(Rp)
          else if (Q.size > Qmax.size) {
            Qmax = Q
          }
          Q = Q - p
        }
        else Qmax
        R = R - p
      }
      Qmax
    }

    expand(g.V)
  }
\end{lstlisting}

\section{Porównanie algorytmów}
\label{algorithm-comparison}
Algorytmy zostały porównane przy użyciu grafów DIMACS\citep{dimacs}. Są to na tyle nietrywialne grafy, że otrzymanie najlepszego znanego wyniku dla danego grafu zajmuje bardzo długi czas. Najlepszy znany wynik dla grafu C125.9\footnote{graf losowy z 125 wierzchołkami i prawdopodobieństwem krawędzi 0.9}  to $w(g)=34$. Aby ten wynik otrzymać algorytm BasicMC potrzebował, aż 5 godzin i 25 minut. Klikę o wielkości 33 znalazł po 15 minutach.

Zdecydowaliśmy się zatem ograniczyć czas działania algorytmów do 90 sekund na każdy problem i porównać wyniki pomijając fakt, że są one dużo gorsze od najlepszych rozwiązań.

Grafy DIMACS można znaleźć w \cite{dimacs}.

\subsection{Warunki przeprowadzonych testów na zbiorze DIMACS}
Każdy test został przeprowadzony w następujących warunkach:

\begin{itemize}
\item Program ma do dyspozycji 2G pamięci
\item Czas mierzony jest dopiero od momentu uruchomienia algorytmu, a nie programu\footnote{dla niektórych problemów grafowych, samo zbudowanie grafu trwa parę minut}
\item Algorytm zwraca kolejne coraz lepsze wyniki. Dla każdego wyniku liczony jest czas jego otrzymania.
\item Końcowym wynikiem jest najlepszy wynik otrzymany w ciągu 90 sekund.\footnote{sytuacja, w której algorytm kończy działanie przed upływem czasu jest obsługiwana, ale w praktyce dla grafów ze zbioru DIMACS nie występuje}
\item Algorytm uruchamiany jest bez rozgrzewania JVM. Każde uruchomienie jest na tyle długie, że rozgrzewanie jest zbędne.
\item Program uruchamiany był na procesorze i7 2.6GHz
\end{itemize}

\subsection{Wyniki testów dla grafów ze zbioru DIMACS}
Na początku przyjrzymy się wynikom otrzymanym dla każdego z grafów, a następnie na czas otrzymania tego wyniku. Wyniki przedstawione zostały na wykresach \ref{fig:dimacs-best-part1} i \ref{fig:dimacs-best-part2}. 

\subsubsection{Najlepsze rozwiązania}
Można zauważyć, że dla każdego grafu algorytm BasicMC daje takie same lub lepsze wyniki przy danych ograniczeniach.

\begin{figure}[H]
  \begin{center}
  \fbox{
    \includegraphics[width=\textwidth]{img/dimacs1.pdf}
  }
  \end{center}
  \caption{Najlepszy wynik dla grafów DIMACS w czasie 90 sekund, część 1}
  \label{fig:dimacs-best-part1}
\end{figure}

\begin{figure}[H]
  \begin{center}
  \fbox{
    \includegraphics[width=\textwidth]{img/dimacs2.pdf}
  }
  \end{center}
  \caption{Najlepszy wynik dla grafów DIMACS w czasie 90 sekund, część 2}
  \label{fig:dimacs-best-part2}
\end{figure}

Uruchomienie algorytmu dla grafu MANN\_a81.clq dla algorytmu Brona-Kerboscha spowodowało błąd braku pamięci. Graf ten ma 3321 wierzchołków i 5 506 380 krawędzi. Test został uruchomiony ponownie z 4G pamięci. Przy ponownym uruchomieniu algorytm wykonał się poprawnie, ale nie zwrócił żadnych wyników - żadna klika nie została znaleziona w czasie 90 sekund działania algorytmu.

\subsubsection{Czas dla najlepszych rozwiązań}
Wyniki przedstawione zostały na wykresach \ref{fig:dimacs-best-time-part1} i \ref{fig:dimacs-best-time-part2}. Na wykresie, dla niektórych grafów widzimy bardzo niskie czasy. Nie oznacza to jedak, że algorytm znalazł optymalne rozwiązanie w krótkim czasie. Oznacza to, że algorytm znalazł jedno rozwiązanie dość szybko, a potem nie mógł znaleźć żadnego lepszego rozwiązania w ciągu kolejnych 90 sekund. Dotyczy to grafów 
p\_hat300\_1.clq, p\_hat1500-1.clq, gen200\_p0.9\_44, brock200\_2.b, DSJC500. W niektórych przypadkach dotyczy to tylko algorytmu Brona-Kerboscha np. dla grafu C500.9.clq algorytm BasicMC znalazł lepsze rozwiązanie przed końcem czasu.


\begin{figure}[H]
  \begin{center}
  \fbox{
    \includegraphics[width=\textwidth]{img/dimacs1czas.pdf}
  }
  \end{center}
  \caption{Czas osiągnięcia wyniku dla grafów DIMACS w czasie 90 sekund, część 1}
  \label{fig:dimacs-best-time-part1}
\end{figure}

\begin{figure}[H]
  \begin{center}
  \fbox{
    \includegraphics[width=\textwidth]{img/dimacs2czas.pdf}
  }
  \end{center}
  \caption{Czas osiągnięcia wyniku dla grafów DIMACS w czasie 90 sekund, część 2}
  \label{fig:dimacs-best-time-part2}
\end{figure}


\subsection{Szybkość znajdowania kolejnych rozwiązań}
Test ten został przeprowadzony na grafie C125.9 z ograniczeniem 20 minut. Pozostałe warunki testowe są takie same. Program uruchomiony został z opcją wyświetlania coraz lepszych wyników. Wykres \ref{fig:ProgressC125.9-20min} przedstawia wyniki tego testu. Można zauważyć kilka ciekawych rzeczy. Po pierwsze algorytm BasicMC zdążył znaleźć lepszy wynik. Najlepszy wynik znaleziony przez algorytm Brona-Kerboscha został znaleziony przez algorytm BasicMC w czasie o rząd wielkości mniejszym. Widać tutaj także regularność - BasicMC znajduje rozwiązania szybciej od algorytmu Brona-Kerboscha dla tego typu grafu.

\begin{figure}[H]
  \begin{center}
  \fbox{
    \includegraphics[width=\textwidth]{img/progress.pdf}
  }
  \end{center}
  \caption{Kolejne wyniki w czasie dla grafu C125.9 z ograniczeniem 20 minut}
  \label{fig:ProgressC125.9-20min}
\end{figure}

\section{Badanie złożoności}

Zależność czasu wykonania algorytmów w stosunku do rozmiaru grafu wejściowego została przedstawiona na \ref{fig:time-complexity}. Widać, że czas wykonania obydwu algorytmów rośnie wykładniczo w stosunku do wielkości grafów wejściowych co jest zgodne z oczekiwaniami. Warto jednak zauważyć, że algorytm BasicMC dla sprawdzonych grafów w rzeczywistości wymaga mniej czasu niż implementacja algorytmu Brona-Kerbosch biblioteki z JGraphT.

\begin{figure}[H]
  \begin{center}
  \fbox{
    \includegraphics[width=\textwidth]{img/czas.png}
  }
  \end{center}
  \caption{Złożoność Czasowa Algorytmów}
  \label{fig:time-complexity}
\end{figure}
\subsection{Złożoność pamięciowa}
\label{memory_complexity}

Zależność pamięciowa algorytmów w stosunku do rozmiaru grafu wejściowego dla niektórych losowo wygenerowanych grafów została przedstawiona na \ref{fig:memory-complexity}. Widać, że zużywana pamiec rośnie wykładniczo w stosunku do wielkości grafów wejściowych co jest zgodne z oczekiwaniami. Punkty odstające są wynikiem działania GC\footnote{skrót od ang. Garbage Collector}.

\begin{figure}[H]
  \begin{center}
  \fbox{
    \includegraphics[width=\textwidth]{img/pamiec.png}
  }
  \end{center}
  \caption{Złożoność Pamięciowa Algorytmów}
  \label{fig:memory-complexity}
\end{figure}

\section{Testy Poprawności Implementacji}
Struktura grafów użytych w testach jest generowana losowo przy każdym uruchomieniu testów.

\subsection{Przypadki trywialne}
W katalogu \textit{test/graphs/} zostały umieszczone testy podstawowych przypadków brzegowych dla zaimplementowanego algorytmu znajdowania największej kliki BasicMC.
Testy sprawdzają czy zachowane są następujące własności grafu:
\begin{itemize}
  \item maksymalna klika grafu pustego powinna mieć wielkość 1
  \item graf pełny ma maksymalną klikę wielkości równej liczbie wierzchołków
  \item cykl ma maksymalną klikę o wielkości 2
  \item graf zbudowany na cięciwach okręgu (circle graph) o liczbie wierzchołków większej niż 4 ma maksymalną klikę o wielkości 3
  \item graf zbudowany na cięciwach okręgu (circle graph) o liczbie wierzchołków równej 4 ma max. klikę o wielkości 4
  \item drzewo ma maksymalną klikę o wielkości 2
  \item graf niespójny mający dwa podgrafy $K_{n_{1}}$ $K_{n_{2}}$ posiada maksymalną klikę wynoszącą $max(n_{1}, n_{2})$
\end{itemize}

\subsection{Przypadek nietrywialny}
Oprócz podstawowych przypadków algorytm został także przetestowany na losowo wygenerowanym grafie z częściowo znaną strukturą. Do losowo wygenerowanego grafu dołączone zostały podgrafy K20, K5, K7. Dla tak powstałego spójnego grafu algorytm prawidłowo znalazł klikę maksymalną o wielkości 20.

\subsection{Testy generatora losowych grafów}
Testy które sprawdzają, że losowo wygenerowany graf jest poprawny:
\begin{itemize}
  \item test spójności grafu - każdy wierzchołek ma conajmniej jednego sąsiada
  \item test na oczekiwaną liczbę krawędzi grafu z zadanym prawdopodobieństwem $p$ - liczba krawędzi przy prawdopodobieństwie p zawiera się w oczekiwanej liczbie krawędzi dla losowego grafu od $(p - \epsilon, p + \epsilon)$
  \item graf ma zadaną liczbę wierzchołków $n$
\end{itemize}

\subsection{Uruchamianie testów}
Aby uruchomić testy należy wykonać \textit{sbt test}\footnote{wymagany jest zatem program sbt http://www.scala-sbt.org/} w głównym katalogu programu. Pełna lista spełnianych asercji została umieszczona w sprawozdaniu \ref{fig:tests}.

\begin{figure}[H]
  \begin{center}
  \fbox{
    \includegraphics[width=\textwidth]{img/tests.png}
  }
  \end{center}
  \caption{Asercje spełniane przez BasicMC}
  \label{fig:tests}
\end{figure}


\section{Wnioski końcowe}
Porównując algorytmy dochodzimy do wniosku, że algorytm BasicMC jest algorytmem lepszym przy danych założeniach testowych. Osiąga lepsze wyniki w mniejszym czasie i wykazuje się mniejszymi wymaganiami pamięciowymi chociażby dlatego, że nie przechowuje wszystkich znalezionych klik, a jedynie najlepsze rozwiązania.

TODO co więcej można tutaj napisać? Jakieś ogólne wnioski na temat projektu, np. ze algorytm w Scali dobrze wygląda? Że jakieś stosunkowo niezłe (nie najlepsze) rozwiązania da się znaleźć w krótkim czasie?

\nocite{*}
\bibliographystyle{plainnat}
\bibliography{bibliography}
\end{document}
